%% 
%% Copyright 2019-2024 Elsevier Ltd
%% 
%% This file is part of the 'CAS Bundle'.
%% --------------------------------------
%% 
%% It may be distributed under the conditions of the LaTeX Project Public
%% License, either version 1.3c of this license or (at your option) any
%% later version.  The latest version of this license is in
%%    http://www.latex-project.org/lppl.txt
%% and version 1.3c or later is part of all distributions of LaTeX
%% version 1999/12/01 or later.
%% 
%% The list of all files belonging to the 'CAS Bundle' is
%% given in the file `manifest.txt'.
%% 
%% Template article for cas-dc documentclass for 
%% double column output.

\documentclass[a4paper,fleqn]{cas-dc}

% If the frontmatter runs over more than one page
% use the longmktitle option.

%\documentclass[a4paper,fleqn,longmktitle]{cas-dc}

% \usepackage[numbers]{natbib}
\usepackage[authoryear]{natbib}
% \usepackage[authoryear,longnamesfirst]{natbib}

\usepackage{microtype}
\usepackage{tabularray}
\usepackage{booktabs}
\usepackage[version=3]{mhchem}
\usepackage[inter-unit-product =\cdot]{siunitx}
\usepackage{stfloats}
\graphicspath{ {./img/} }
%%%Author macros
\def\tsc#1{\csdef{#1}{\textsc{\lowercase{#1}}\xspace}}
\tsc{WGM}
\tsc{QE}
%%%

% Uncomment and use as if needed
%\newtheorem{theorem}{Theorem}
%\newtheorem{lemma}[theorem]{Lemma}
%\newdefinition{rmk}{Remark}
%\newproof{pf}{Proof}
%\newproof{pot}{Proof of Theorem \ref{thm}}

\begin{document}
\let\WriteBookmarks\relax
\def\floatpagepagefraction{1}
\def\textpagefraction{.001}

% Short title
\shorttitle{Data-Driven Modelling of Cyclic Voltammetry in Upcycled Carbon Supercapacitors}    

% Short author
\shortauthors{M. A. Sandoval-Riofrio and K. Jayathunge}  

% Main title of the paper
\title [mode = title]{Performance of upcycled tyre-derived carbon in ethaline-glycol ionic electrolytes for sustainable supercapacitors: modelling cyclic voltammetry using standard and machine-learning methods.}  

% Title footnote mark
% eg: \tnotemark[1]
% \tnotemark[1] 

% Title footnote 1.
% eg: \tnotetext[1]{Title footnote text}
% \tnotetext[1]{} 

% First author
%
% Options: Use if required
\author[1]{Maria A. Sandoval-Riofrio}[
  orcid=0000-0003-0680-4539
]
% Corresponding author indication
\cormark[1]
% Footnote of the first author
\fnmark[1]
% Email id of the first author
\ead{msandoval2@bournemouth.ac.uk}
% URL of the first author
\ead[url]{https://www.linkedin.com/in/maria-a-sandoval/}


% Credit authorship
% eg: \credit{Conceptualization of this study, Methodology, Software}
\credit{Conceptualization of study, physical experiments, analysis}


\author[1]{Kavisha Jayathunge}[
  orcid=0009-0006-5162-5731
]

% Footnote of the second author
\fnmark[2]

% Email id of the second author
\ead{kjayathunge@bournemouth.ac.uk}

% URL of the second author
\ead[url]{https://staffprofiles.bournemouth.ac.uk/display/kjayathunge}

% Credit authorship
\credit{Data processing, Machine learning modelling, analysis}

% Address/affiliation
\affiliation[1]{organization={Bournemouth University},
            addressline={Talbot Campus, Fern Barrow}, 
            city={Poole},
            citysep={}, % Uncomment if no comma needed between city and postcode
            postcode={BH12 5BB}, 
            state={Dorset},
            country={UK}}

% Corresponding author text
\cortext[1]{Corresponding author}

% Footnote text
\fntext[1]{School of Computing and Engineering}
\fntext[2]{National Centre for Computer Animation}

% For a title note without a number/mark
%\nonumnote{}

% Here goes the abstract
\begin{abstract}
  This study explores the development of sustainable supercapacitors using electrochemically activated carbon derived from plastic and tyre waste as electrode materials and ethaline, a deep eutectic solvent, as a green electrolyte. The electrochemical molten salts (EMS) process was employed to activate black carbon, a waste byproduct from tyre recycling, resulting in significant enhancements in carbon properties, including a 39.4\% increase in BET surface area (from \SI{43.10}{\meter\squared\per\gram} to \SI{60.09}{\meter\squared\per\gram}) and improved pore volume and size distribution. These findings establish the EMS process as a scalable and effective method for producing high-performance porous carbon materials.

  Supercapacitors assembled with EMS-activated carbon and ethaline exhibited remarkable electrochemical performance, achieving a specific capacitance by GDC of \SI{92.2}{\farad\per\gram}  at \SI{0.4}{\ampere\per\gram}, and \SI{110.16}{\farad\per\gram} by cyclic voltammetry, surpassing many conventional aqueous and organic systems. Combined CV and EIS analyses revealed primarily capacitive behaviour with notable contributions from diffusion processes in the activated carbon by EMS. Multiple machine learning model configurations for predicting specific capacitance showed that incorporating an area-based loss consistently improves accuracy and stability across runs. The best-performing model achieves an average error of 3.75\%, compared with a baseline error of 15.49\%. Analysis of the model’s learned weights indicates that electrolyte temperature in the EMS process is the dominant factor influencing BC6 energy storage performance  measured by specific capacitance, followed closely by active mass and electrolysis voltage.

  These results highlight the potential of integrating waste-derived carbon materials with environmentally friendly electrolytes to create efficient, sustainable energy storage systems, while machine-learning-based performance assessment supports reproducibility and reduces experimental redundancy, minimising material, and energy waste during manufacturing.
  
\end{abstract}

% Use if graphical abstract is present
%\begin{graphicalabstract}
%\includegraphics{}
%\end{graphicalabstract}

\begin{highlights}
\item 
\item 
\item 
\end{highlights}

% Keywords
% Each keyword is seperated by \sep
\begin{keywords}
  Supercapacitors \sep Machine learning \sep Electrochemical molten salts \sep Black carbon
\end{keywords}

\maketitle

\section{Introduction}
In an era where environmental sustainability is of primary importance for anthropogenic activities at all scales, it is crucial to find innovative solutions to reduce energy consumption and minimise carbon footprint. Capacitors are energy storage components that play a vital role in various systems but also offer important environmental benefits and technical advantages compared to other energy storage devices like batteries1-4. In recent years, more efficient supercapacitors (SCs) have been developed and are leading the market as they are fulfilling the gaps in the applications where batteries present limitations including thermal stability, extended long life, and maintenance-free operation1,5-7.  In general, electrochemical energy storage devices, such as batteries or electrochemical capacitors, store energy by converting electrical energy into chemical energy, and active materials influence the cycling stability and lifespan of the device8. Porous materials for SCs play a key role in the energy storage performance. An electric double-layer capacitor (EDLC) offers higher capacitance and energy density than an electrolytic capacitor. The efficiency of a supercapacitor is influenced mainly by the selection of electrode material. One of the most remarkable characteristics of a high-performance capacitor is possessing a large surface area electrode material that also has considerable compatibility with the selected electrolyte6,9. Current SC electrodes employ carbon-based materials as active materials with diverse surface areas, for instance, activated carbon (AC) possesses a surface area between ~575 and \SI{3000}{\meter\squared\per\gram}10-12, functionalised carbon nanotubes (CNTs) in ranges of ~50 and \SI{1315}{\meter\squared\per\gram}13-15 and the most popular for energy storage purposes due to their exceptional conductivity properties is graphene16,17 (surface area of ~\SI{2630}{\meter\squared\per\gram}). Charcoal, conducting polymers, and metal oxides have been also employed as electrode materials, nonetheless, the synthesis methods and the combination of them with carbon-based materials require innovative techniques that allow the production of efficient SCs utilising active materials sourced from alternative origins or waste streams18.   By integrating these sustainable materials, energy storage technologies can become more competitive, adaptable to specific market needs, and scalable for large-scale applications19.

Conversely, the high demand for advanced electrode materials requires new technology and feedstocks which is attracting particular attention due to the drawbacks of the mining of raw materials and environmental limitations of existing technologies to synthesise cathode active materials (CAM) along with composites with enhanced electrochemical capacitance properties1,20. The synthesis and activation of carbon-based porous materials have attracted considerable interest due to their potential as highly effective electrode materials in supercapacitor design. They possess outstanding mechanical strength, large specific surface area, and, most importantly, superior electrical conductivity21-23. Nevertheless, traditional methods for producing electrodes are often complex and resource-demanding, frequently involving scarce and hazardous substances23. 
This leads to additional expenditure and escalates in costs but also results in significant environmental consequences. Additionally, the current methods of carbon activation and the augmentation of specific surface area like pyrolysis, have been utilised in the last decades24. Pyrolysis involves using an inert environment to shield organic materials from oxidation and requires high temperatures to modify the morphology and physiochemical properties of carbon. In terms of sustainability, it is remarkable to introduce less energy-consuming technologies and feedstock circularity concepts within advanced materials manufacturing25. Within this context, plastic waste seems a feasible carbon source especially if it represents a major global plastic pollutant like tyre waste. Around 24 million tyres are sold every year, and their disposal leads to a significant environmental impact26,27. Tyres possess a complex mix of materials including reinforced rubber and additives, making recycling challenging, particularly when converting tyre carbon into commercial products like black carbon that must meet standards for market competitiveness, commercial viability and scalability26,28,29. 

However, these conventional approaches are being increasingly challenged by new methodologies that prioritise process integration, resource efficiency, and reduced environmental impact. In this study, electrochemical molten salts are discussed as an alternative to increasing the surface area and nanoporous of tyre pyrolysed carbon23,25,30. Electrochemical molten salts (EMS) technology has gained significant importance in recent years due to its versatility with various salts, temperatures, and voltages. On the other hand, EMS is widely applied for materials processing such as steel, non-ferrous and rare earth metal industries for refining and precision heat treatment processes limited by thermodynamic or kinetic factors31,32. In the last years, renewable energy infrastructure, lithium batteries, supercapacitors, hydrogen fuel cells, semiconductors, and other pioneering green and carbon-reduced technologies have been linked to developing new metal materials. EMS has become a popular technology due to its potential to recover critical metals, hydrogen obtention, and \ce{CO2} capture by electrolysis31,33-37.  Under this scheme, the molten salt \ce{CO2} electrochemical capture has demonstrated a promising scenario for carbon-based materials manufacturers to produce advanced energy materials aligned to the current trends to relieve global warming. The term carbon black (CB) is often misused and confused with black carbon (BC). CB represents raw materials commercialised at a large scale and is part of other value and supply chains. Black carbon (BC) is not produced at a large scale to be commercialised and the coined term BC refers to fine particles normally sourced as an unwanted byproduct/residues of other manufacturing processes such as pyrolysis or other commercially manufactured products including fossil fuels (char, tar, coal-tar), plastic, rubber amongst others38. In this work, a novel method to convert pyrolysed plastic from tyre waste (BC) is proposed as a novel activation method to promote nanoporous formation by the electrochemical molten salt process. Adopting a sustainable technology approach, the electrochemical performance of the obtained nanoporous carbon was tested as cathode-anode active materials and their interaction with green electrolytes was evaluated in supercapacitors. 

Machine learning (ML) supports a data-driven role in this work, focusing on the influence of EMS process conditions  and how they influence on the electrochemical  energy-storage behaviour of the resulting materials through specific capacitance evaluation. Electrochemical systems are inherently multi-parameter and nonlinear, with electrode microstructure, electrolyte composition, and EMS-activation conditions interacting in ways that are difficult to disentangle through conventional parametric studies alone39-41. ML is increasingly used in electrochemical energy-storage research for performance assessment, sensitivity analysis, and reduction of experimental redundancy, enabling researchers to map complex input-output relationships and identify the most influential variables without exhaustive trial-and-error experimentation at scaling-up stage40,42-44. In this study, ML models are trained on experimentally obtained EMS-activated carbon data to correlate material activation process signifiers (e.g., temperature, retention time and voltage), with key electrochemical metrics performed in a symmetric two-electrode supercapacitor configuration. Since both electrodes are identical and contribute equally to charge storage, the specific capacitance was calculated by normalizing the capacitance to the active mass of a single electrode (half-cell basis)45-47. Accordingly, the mass used in the ML analysis corresponds to half of the total active mass of the device, thereby supporting mechanistic interpretation while remaining grounded in empirical measurements.

ML can also contribute to faster scale-up and manufacturing of EMS-activated carbons by guiding the design of efficient experimentation and repeatability experiments on SCs manufacturing48,49. Recent studies demonstrate that ML-assisted optimisation of synthesis and operating parameters can significantly reduce the number of required experiments while improving target electrochemical performance, which accelerates the transition from lab-scale to pilot- and industrial-scale production41,50,51. By combining ML-based performance prediction with sensitivity analysis, R\&D on SCs can prioritise robust, high-yield regions of parameter space, minimise costly over-testing, reduce the waste generated at laboratory scale and support ``smart manufacturing'' strategies for electrochemical materials. In this work, ML is limited to performance characterisation and post-synthesis analysis of the tire wasted-derived carbon and is not used for real-time control of the EMS activation process. 

\section{Experimental section}
\subsection{Electrochemical activation of black carbon}
The EMS procedure was conducted within a sealed electrochemical cell (Nabertherm Tube Furnace RT 50) utilising a mixture of \SI{50}{\gram} of \ce{LiCl} and \SI{50}{\gram} of \ce{KCl} salts as electrolytes (Sigma Aldrich, 99\%) in a 1:1 molar ratio. The molten salts to form the eutectic mixture, molar ratio and operation parameters were determined based on prior experimentation and based on literature  52-55. Optimal operational conditions were identified under an \ce{Ar} atmosphere as follows: 2 hours, a constant current of \SI{0.2}{\ampere} at \SI{2.5}{\volt}, and a temperature of \SI{700}{\degreeCelsius}, as illustrated in Figure 1. For electrochemical conversion, the cathode employed was a graphite rod of \SI{1}{\milli\meter} diameter (graphite, 99.95\%). The anode comprised a stainless-steel rod with approximately 0.3\% nickel content combined with a black carbon (BC) sample that was pre-packed into carbon fibre in a cylindrical shape. The activated BC (BC6) was characterised after EMS activation and used to assemble SC electrodes. 
[INSERT FIGURE 1: a) Computer-aided scheme of EMS reactor. b) Best parameters for black carbon activation and c) EMS reactor set-up. ]

\subsection{Fabrication of activated carbon electrodes for the SCs}
Electrodes were prepared by mixing the BC and activated carbon post-EMS BC6 as carbon active material (95\% w/w) with 5 \% w/w polyvinylidene fluoride (PVDF) binder (Sigma-Aldrich, 10\% suspension in N-methyl-pyrrolidone, NMP) in a slurry with NMP. The slurry was stirred for 6 hours at 3000 rpm and subsequently coated on aluminium foil (\SI{10}{\micro\meter} thickness) and dried at \SI{70}{\degreeCelsius} for 12 hours. The electrodes have an area of ~\SI{1}{\centi\meter\squared} with mass loadings from around ~\SI{0.19}{\milli\gram\per\centi\meter\squared} of the total active carbon material. The electrodes were cast by doctor blade coating (Figure 2) dried at \SI{70}{\degreeCelsius} for around \SI{24}{\hour} and subsequently were sandwiched using a cellulose paper filter as a separator. 
[INSERT FIGURE 2: General scheme of the electrode design and preparation through doctor blade coating. ]


\subsection{Characterisation of materials}
The sample BC and post-EMS activated carbon (BC6) were prepared using a gold coating (~\SI{5}{\nano\meter}) employing scanning electron microscopy (SEM) using JEOL JSM-5800LV at \SI{15}{\kilo\volt} coupled with energy dispersive X-ray (SEM/EDX) spectroscopy for determining the elemental composition. The spectroscopy analysis by FTIR and Raman and their surface functional groups of the samples were analysed by an Agilent Cary 630 Spectrometer FTIR (Fourier Transform Infrared Spectroscopy) within the range of 400-\SI{4000}{\per\centi\meter}. The Raman spectra were estimated using JY Horiba Lab RAM HR using an excitation wavelength of \SI{532}{\nano\meter} and laser power of \SI{28}{\micro\watt}. The particle size distribution of BC samples in DMF was investigated using DLS (Malvern Zetasizer, Nano ZS Series) whilst the surface area was estimated by the multi-point Brunauer, Emmett and Teller (BET) method and complementing the BET analysis the BJH (Barrett, Joyner, and Halenda) method using Quantachrome Nova 1200e equipment. The crystal structure of BC samples was evaluated using a Siemens D5000 X-ray Powder Diffraction (XRD) System with \ce{Cu} K$\alpha$ ($\lambda$ = \SI{0.154}{\nano\meter}). The XRD analysis was carried out under irradiation in the 2$\theta$ range of \SI{20}{\degree} to \SI{80}{\degree}. The chemical state of the BCo and BC6 samples were analysed using X-ray photoelectron spectroscopy XPS (Thermo-Scientific, EXCALAB Qxi).  

\subsection{Electrochemical measurements}
Electrochemical performance tests using cyclic voltammetry (CV) were run on assembled supercapacitors (SCs) cells with active material weights within the range of \SI{2.1}{\milli\gram} to \SI{3.9}{\milli\gram}. Initial CV tests were run to evaluate the interaction of the black carbon (Bco) prior activation using choline chloride-based DESs (deep eutectic solvents) and a commonly used water-based electrolyte 0.1 M \ce{Na2SO4} (Aldrich, 99\%), I (Aldrich, 99\%)  and DESs. The electrolytes choline-chloride based (hydrogen bond acceptor HBA) were prepared combining them with hydrogen bond donors (HBD). The electrolytes tested were ethaline 200 (ethylene glycol, Aldrich, 99.8\%), oxaline (oxalic acid, Aldrich 98\%) and glyceline (glycerine, Aldrich, 99\%) were prepared according to the methods56 
Electrochemical measurements were conducted using a symmetric two-electrode supercapacitor setup. Given that both electrodes are identical and exhibit equivalent charge storage behaviour, the specific capacitance was determined by normalizing the measured capacitance to the active mass of one electrode (single electrode basis)57. The electrochemical experimental data are used for calculating the specific capacitance according to the following equation based on the current response and the scan rate:

\begin{equation}
  \label{eq:specific_cap}
  C_{sp} = \frac{\int I(V) \,dV}{ 2 \cdot \nu \cdot m \cdot \Delta V}
\end{equation}

where $\int I(V) \,dV$ is the integral of the current $I$ (\unit{\ampere}) over the voltage $V$ (\unit{\volt}), $E$ is the potential, $\nu$ is the scanrate (\unit{\volt\per\second}), m is the total mass of the active material in one electrode (\unit{\gram}), and $\Delta V$ is the potential window (\unit{\volt}).

A second round of CV was carried out to evaluate the electrochemical performance of the electrolyte against the activated carbon sample BC6 from the post-EMS activation process. The assembled supercapacitor with BC6 sample as active material was compared using ethaline 200 with other two electrolytes used in commercial electrolytes such as acetonitrile (Aldrich, 99\%) and 0.6 M \ce{KOH} (Aldrich, >85\%). 
The electrochemical performances were evaluated in a two-electrode system (symmetric) using cyclic voltammetry (CV), galvanostatic charge-discharge (GCD) and electrochemical impedance spectroscopy (EIS) using an IviumStat electrochemical interface station. The electrochemical impedance spectroscopy (EIS) analysis was carried out by employing an amplitude of \SI{5}{\milli\volt} over the frequency range from \SI{0.1}{\hertz} to \SI{100}{\kilo\hertz}. 

\section{Results and discussion}
\subsection{Materials characterisation}
\subsubsection{SEM images and EDS of surface elements of BC}
[INSERT FIGURE 3: (a) Black carbon (Bco) prior activation by EMS. (b)-(c) Activated carbon by EMS (BC6). (d) Approximate pore diameter (\SI{200}{\nano\meter}) of BC6 sample. (c)-(d) EDS-SEM of Bco for \ce{Si}-element and \ce{S}-element. I-(f) EDS-SEM of BC6 (post activation by EMS) for \ce{Si}-element and \ce{Si}-element and \ce{S}-element.]
Figure 3 displays scanning electron microscope (SEM) images of black carbon (Bco) and activated carbon powders before and after electrochemical molten salt (EMS) activation. A substantial alteration in surface structure is evident, with increased roughness and porosity in specific areas. Elemental analysis using EDS-SEM of the original Bco in Figure 3I revealed the existence of silicon (17\%) and in Figure 3(d) sulphur (11\%) distributed throughout the pyrolysed carbon particles. Sulphur is a prevalent contaminant in tyre waste, while silica is a widely utilised filler in rubber, providing cost-effective reinforcement due to its non-polar functional groups and hydrogen bonding with the elastomer58. Despite that sulphur is not the most desirable compound in a precursor of carbon-based materials, some literature has reported that sulphur can act as a highly efficient material for anodes in sodium-ion batteries and trigger more efficient faradaic redox interactions happening in \ce{S}-doped carbonaceous electrodes, leading to increased pseudocapacitance in supercapacitors59,60. In the BC6 sample, the silica content rose to 27\% in Figure 3I, while the sulphur content diminished to 6\% in Figure 3(f). Molten salt electrochemistry decreased the sulphur content in BC6 by 54\%, presenting an alternative approach to desulfurisation. 

\subsubsection{BET analysis and DLS particle size distribution}
[INSERT FIGURE 4: Black carbon (Bco) and activated carbon (BC6) surface analysis by: (a)-(b) BET adsorption isotherms. C) DLS particle size distribution and d) Pore size distribution curves.]
Figure 4 illustrates the application of the BET equation to determine specific surface area using a multilayer adsorption model. The BET-estimated area for the Bco sample is \SI{43.1}{\meter\squared\per\gram}, while for BC6 post-EMS activation, it increases to \SI{60.09}{\meter\squared\per\gram}. This represents a 39.4\% rise in apparent surface area, suggesting the formation of new porous structures on the carbon's surface61. The BJH method was employed as a supplementary technique to calculate an average pore volume of \SI{0.23}{\centi\meter\cubed\per\gram} and a pore size of \SI{12.7}{\nano\meter} (127 \AA) for Bco, while for BC6, these values were \SI{0.35}{\centi\meter\cubed\per\gram} and \SI{1.96}{\nano\meter} (19.6 \AA), respectively. The isotherms are of type II with a type H4 hysteresis loop in Figure 4 (a)-(b), characteristic of monocoat (monolayer)-multilayer adsorption in a network with mesoporous and microporous features typical of silt pores3,62,63. EMS activation method modified on the average value of the size of pores of the newly formed nanoporous, impacting the total surface area of BC6 available for adsorption and most notably, the size of molecules that can diffuse into the solid. Hence, the development of a suitable pore structure will affect the potential application of activated carbon 64,65. 
The type H4 hysteresis loop is also an indicator of meso-microporous structures in carbon materials like those found in mesoporous zeolites and silicas66. Adsorption sites play a key role in the SCs charging process as the accumulation of the electroactive species occupies the active sites of the carbon material whilst the surface diffusion of the adsorbed species will be distributed in the high surface area of carbon. Sulphur atoms create larger interplanar distances in the micro and nanoporous structures. At the same time, sulphur is renowned for its high reactivity among heteroatom dopants due to its unpaired electrons (a polarizable character from a wider bandgap from an electron-withdrawing group) due to the subtle difference in electronegativity with carbon59,67,68. Electrochemical activation of recycled tyre waste into high-surface-area activated carbon offers a sustainable alternative to silica in carbon-based anodes. EMS process addresses the environmental issues associated with pyrolyzed carbon, with an increased amount of silica (27\% in BC6). By incorporating nanoscale silica into the activated carbon, this method enhances electrode performance by reducing stress, strain, and cracking, leading to increased reversible capacity 69,70. The repurposing of pyrolyzed tyre waste minimises environmental harm, fosters sustainability, and aligns with the principles of a circular economy. By transforming discarded materials into valuable resources, the EMS approach reduces waste and creates new economic opportunities 71. 
To determine the particle size of the carbon black (BC) particles before and after the electrochemical modification (EMS) process, dynamic light scattering (DLS) analysis was conducted. The results depicted in Figure 4I indicate that the activated carbon particles have an average diameter of \SI{300}{\nano\meter}. However, the correlation function suggests a slightly faster decay, which is indicative of smaller particles exhibiting more rapid fluctuations in scattered light and a more rapid decline in the correlation function over time. Although there may not be a substantial change in particle size following EMS activation, some studies have observed a sharper peak, suggesting the potential presence of aggregated particles, a common occurrence in activation methods like carbonization or physical activation (e.g., steam activation) 66,72. In Figure 4(d) the pore size distribution plot characterizes the porosity of materials by showing the volume distribution of pores across different sizes by (BJH) method from nitrogen adsorption data. The Bco sample exhibits a dominant peak at ~\SI{2.5}{\nano\meter}, reflecting smaller mesopores, whereas the BC6 sample shifts to larger pore sizes (~3-\SI{4}{\nano\meter}) with a higher overall pore volume, aligning with its greater nitrogen adsorption capacity. These differences suggest structural modifications in BC6, making it potentially more suitable for applications requiring enhanced porosity, such as catalysis and adsorption that are also ideal for long-term energy storage because they share critical properties such as high surface area, tunable porosity, and chemical stability73. High surface area is essential for both catalysis, where it increases the number of active sites for reactions, and energy storage, where it enhances ion exchange and electron transfer, improving energy density and efficiency 73,74. Likewise, tuneable porosity cause by the EMS activation, as in the anode, the carbon atoms interact with various oxide ions adsorbed on the surface of the carbon material, resulting in the production of \ce{CO} gas (Equations~\ref{eq:co_1}--~\ref{eq:co_2}). The generation of both the ions and CO contributes to the formation of pores in the carbon surface, facilitating the efficient molecule trapping and release processes enabling fast ion diffusion and electrolyte access in batteries and supercapacitors74,75. 

\begin{align}
  \ce{C + CO2 &-> 2CO} \label{eq:co_1} \\
  \ce{C + H20 &-> CO + H2} \label{eq:co_2}
\end{align}

DLS assesses the size of particles suspended in a liquid by monitoring their random movement 76. The hydrodynamic size refers to the diameter of a hypothetical sphere that diffuses at the same rate as the particle of interest, with its intensity directly related to the particle's size 77. Mechanisms to get control over the forces that cause nanoparticles to cluster in a suspension are vital, as the system's behaviour can be influenced by electrostatic interactions, external fields, solvent characteristics, shape, and surface properties 78.  

\subsubsection{Analysis by FTIR, Raman, XRD and XPS spectroscopy}
[INSERT FIGURE 5:(a) Raman spectrum of black carbon (Bco) and post activation by EMS (BC6). (b) FTIR spectrum of Bco and BC6. C) XRD pattern of Bco and activated carbon (BC6) after EMS. D) XPS analysis of BC6 sample (C1s, O1s, C KLL and Si2p spectra).]

Raman spectroscopy was used to analyze the carbon structures of the samples, as illustrated in Figure 5a). The prominent D peak, located at approximately \SI{1320}{\per\centi\meter}, is a distinctive feature of carbonaceous materials with a disordered crystalline/amorphous sp3 structure, observable in both BCO and BC6 79,80. 
Conversely, in BC6, the G band at \SI{1580}{\per\centi\meter} arises from the stretching of sp2 carbon atoms and reflects the degree of graphitization. This graphitization, characterised by cyclic and linear carbon structures, creates a porous matrix that can effectively retain and stabilize OPCM during phase transitions. The BC6 Id/Ig ratio of 0.83 indicates a high level of graphitization, which is further supported by an increase in surface area as demonstrated by BET analysis81. In contrast, the Bco Id/Ig ratio of 0 suggests a disordered and defective carbon structure, similar to that observed in BC6. It is reasonable to conclude that the EMS process also contributes to the formation of highly porous structures with ordered sp2 graphitic carbon, resembling graphite and other carbon-based materials known for their high thermal conductivity and low density  86-88.

The Fourier-transform infrared spectroscopy (FTIR) spectra in Figure 5b) show the surface functional groups of carbon black (BC) before and after electrochemical modification (EMS). The broad peak observed at approximately 3450 to \SI{3700}{\per\centi\meter} can be credited to carbon-hydrogen (\ce{C-H}) and oxygen-hydrogen (\ce{O-H}) functional groups. The peak at \SI{3450}{\per\centi\meter} is a result of the stretching vibration of oxygen-hydrogen bonds from surface-adsorbed water86. The broad peaks observed between approximately 1200 and \SI{1651}{\per\centi\meter} are associated with the stretching vibrations of the carbon-oxygen (\ce{C=O}) and carbon-carbon (\ce{C-C}) functional groups frequently found in tyre rubber waste and commercial carbon black. A broader peak is evident for Bco, while a more defined peak appears at \SI{1651}{\per\centi\meter} for BC6, indicating the presence of the \ce{C=O} group. The peaks within this range are attributed to the stretching frequency of the \ce{C=O} bond, which was also intensified for BC6, suggesting that the oxygen content on the surface of BC6 is higher than that of Bco before EMS modification 85,87. The group of peaks located between \SI{1970}{\per\centi\meter}, \SI{2026}{\per\centi\meter}, and \SI{2165}{\per\centi\meter} are due to the stretching vibrations of the (\ce{C#C}) group of carbon atoms in alkanes, alkyl groups, alkynes, and (\ce{N=C=S}, \ce{SC#N}) groups in thiocyanates. The peak at \SI{2165}{\per\centi\meter} is more prominent in BC6, resembling the characteristics of commercial carbon black N330 added to rubber, which increases the surface area and improves properties such as tear resistance, abrasion resistance, and elasticity in tyres, as documented in the literature. Conversely, the peak present at \SI{750}{\per\centi\meter} corresponds to the \ce{C-S} bond in the Bco sample but is absent in BC6, indicating the removal of sulfur after EMS, similar to the peaks at ~\SI{1298}{\per\centi\meter} and ~\SI{1450}{\per\centi\meter}  associated with the CH3 stretching and methylene \ce{-CH2} group bending 88. The lack of this peak directly correlates with the thermal oxidation of Bco and, consequently, the reduction of elastomer structural groups. The decrease in \ce{-CH2} groups is directly linked to the breakage of \ce{C-S} and \ce{S-S} bonds, indicating an evulcanization effect 89.  Studies related to the role of \ce{C-S} have reported high-performance capacitive energy storage in \ce{Li}-ion batteries and supercapacitors. In general terms, carbon materials posses non-polar \ce{C-C}/\ce{C=C} bonds and adding dopants like \ce{S}, \ce{N} or other functional groups on carbon surface enhanced surface polarity and materials with higher porosity are also favorable to capturing sulphur atoms90-92. On the other hands, silica was confirmed by the presence of a slightly elevated siloxane (\ce{Si-O-Si}) peak at \SI{1100}{\per\centi\meter} in both BC6 and Bco. Additionally, a peak attributed to \ce{Si-O} bond stretching and silanol (\ce{Si-OH}) was observed at \SI{950}{\per\centi\meter} in Bco 93,94. It is remarkable to mention that the peak associated with \ce{Si-O} groups is more prominent in BC6, which aligns with the increased silica content observed in the SEM-EDS analysis of the sample processed under EMS conditions.


XRD analysis shows in Figure 5I that both pyrolysis (Bco) and activation by EMS modified the crystalline structure of black carbon (BC6) derived from waste tyres. The narrower and taller (002) peak, indicative of enhanced graphitization, was observed in BC6 samples treated at \SI{700}{\degreeCelsius} and \SI{2.8}{\volt} for 2 hours during the EMS. The intensities of the (002), (101), and (004) diffraction peaks increased substantially in BC6 compared to Bco, implying a more organized arrangement of aromatic layers, a hallmark of carbon-based materials with carbon content exceeding 81\% wt., such as coal95, and those activated using both, heat treatments and enhanced by the electrochemical reactions 96,97,98. The increased stacking of carbon can be attributed to the breakdown of aliphatic chains under heating conditions, which facilitates the condensation of aromatic rings and leads to their gradual expansion 87. Furthermore, the presence of sulfur within the carbon matrix of the pyrolyzed Bco samples contributes to the graphitization of carbon in BC6 when treated with chloride molten salt at temperatures below \SI{700}{\degreeCelsius}(~\SI{675}{\degreeCelsius}). This makes the EMS activation method more energy-efficient than other thermal activation methods, as it does not require excessively high temperatures to promote the formation of crystallized graphite nanostructures 99. 100.

XPS analysis spectrum shown in Figure 5(d) confirms the presence of carbon I, silicon (\ce{Si}), and oxygen (\ce{O}). C1s spectrum with binding energy at \SI{282}{\electronvolt} corresponding to metal carbide, in BC6 can be attributed to the development of \ce{Si-C} bond. At the same time the presence of \ce{Si} 2p and C1s spectra with the binding energy of ~\SI{100}{\electronvolt} and \SI{282}{\electronvolt} associated with the vulcanized \ce{Si} and \ce{C} atoms of the \ce{SiC1}01,102. Additionally, low-intensity peaks can be detected at ~\SI{260}{\electronvolt} corresponding to a \ce{C} KLL Auger transition due to the ejection of a core-level electron (from the 1s orbital of carbon, or the K-shell of an Auger electron) by the incident X-rays, in this case, it represents a sp2-hybridized carbon (\ce{C} single bond \ce{C} bond)?? that also suggesting some oxygenated functional groups from the activation process102,103. The O1s spectrum exhibits a significant peak at \SI{529}{\electronvolt}. The results of the C1s and O1s peaks suggested a \ce{C=O} particularly when enhanced by molten salt activation104,105.

\subsection{Electrochemical performance}
The electrochemical properties of the electrodes prepared for each supercapacitor (SC) cell assembled with Bco and BC6 electrodes and their interaction with DESs based electrolytes were studied by cyclic voltammetry (CV) as is depicted in Figure 6. The electrochemical experimental data were used to estimate the specific capacitance using Equation~\ref{eq:specific_cap}. 

[INSERT FIGURE 6: A) CV of Bco (black carbon) with water-based electrolytes (Na2SO4) and DESs (ethaline, glyceline, oxaline). B) CV of BC6 (activate carbon) against I, \ce{KOH} and ethaline electrolytes. C) CV of BC6 against ethaline 5 cycles (red), 150 cycles (blue), 1000 cycles (green) and after 10 000 cycles (yellow). D) Scan rates of ethaline with BC6 at \SI{5}{\milli\volt\per\second}, \SI{10}{\milli\volt\per\second}, \SI{50}{\milli\volt\per\second} and \SI{100}{\milli\volt\per\second}.]

In Figure 6 can be observed the cyclic voltammetry of the assembled electrodes in supercapacitor cells at a scanning rate of \SI{100}{\milli\volt\per\second} in a potential range of 0 to \SI{1}{\volt}. The supercapacitors with Bco as active material (prior activation) were evaluated to estimate the interaction with different types of water-based electrolytes (0.1 M \ce{Na2SO4}) and DESs (ethaline 200, oxaline and glyceline) for Bco (Figure 6a). The specific capacitances are shown in Table~\ref{tab:specific_capacitances}. 

\begin{table}
\caption{Specific capacitance estimated by CV on Bco (black carbon)}
\label{tab:specific_capacitances}
\small
\centering
\begin{tblr}{
  hline{1-2} = {-}{},
  colsep = 7pt
}
Electrolyte &  {Specific\\capacitance (\unit{\farad\per\gram})} \\
\ce{Na2SO4} &  13                                           \\
Oxaline     &  3.44                                         \\
Glyceline   &  2.15                                         \\
Ethaline    &  16.33                                        \\
\end{tblr}
\end{table}

The cyclic voltammetry experiments revealed that the BC exhibited conductive behaviour when tested with conventional electrolytes like \ce{Na2SO4}, as well as eco-friendly alternatives known as deep eutectic solvents (DES). These green electrolytes offer characteristics that can enhance the energy storage capabilities of the supercapacitor106. 
Water-based electrolytes offer several advantages for supercapacitors, with \ce{Na2SO4} being one of the most commonly used options. It provides a neutral pH environment, which helps minimise corrosion and extends the lifespan of both the electrodes and current collectors. Moreover, research has shown that \ce{Na2SO4} can operate within a voltage range of \SI{1.23}{\volt} to \SI{1.6}{\volt}, surpassing the typical \SI{1.2}{\volt} limit imposed by water electrolysis. This ability to function at higher voltages contributes to improved energy density while maintaining the stability of both the electrolyte and the electrodes 107-109. Despite its high ionic conductivity, minimising the internal resistance in the supercapacitor, considering the interaction with the black carbon, \ce{Na2SO4} is conductive as it can be observed but it tends to be competitive with other aqueous systems, helping supercapacitors achieve excellent cycling stability and efficiency. The nature of the black carbon from plastic waste, suggests that the Bco electrode doesn't possess the surface wettability required to promote the charge-transfer process and electrode-electrolyte interactions110,111.  Building on this, a DES solvent known as ethaline 200, made from a 1:1 ratio of choline chloride and ethylene glycol, was utilized alongside oxaline and glyceline. Deep eutectic solvents (DESs) are defined as eutectic mixtures composed of Lewis or Br\o nsted acids and bases, typically in the form of organic salt solutions. These solutions often include choline chloride (a quaternary ammonium salt) combined with a hydrogen bond donor (HBD) molecule, such as ethylene glycol, urea, or glycerine, in a specific molar ratio. This unique pairing creates a mixture with a eutectic melting point significantly lower than the melting points of the individual components 112. 

Ethaline is a mixture of quaternary ammonium salts mixed with metal salts or hydrogen bond donors (HBDs) choline chloride normally and ethylene glycol which have been applied in many areas including electrodeposition of metals and alloys. Ethaline provides several advantages such as high electrochemical stability, low volatility, and wide potential windows. DES is promising as an alternative to aqueous and organic electrolytes, especially in high-performance supercapacitors113. Choline chloride is considered a very low acute oral toxic which makes DESs easy to prepare, non-volatile, biodegradable, insensitive to water, and recyclable, all of which make their use in large-scale applications highly beneficial.114-116. Ethaline was selected as the electrolyte for further studies on BC6 (activated carbon by EMS), as it exhibits a more stable and behaviour close to the ideal rectangular shape of voltammograms also with largest area which means the formation of an electric double layer117. 

In contrast, oxaline and glyceline were prepared following the method outlined in118. Both exhibit higher viscosities compared to other organic-based electrolytes like ethanol or acetonitrile; however, they also demonstrate greater conductivity due to their relatively lower viscosities. Oxaline-based DESs, with conductivities ranging from 1 to \SI{10}{\milli\siemens\per\centi\meter}, experience slower ion diffusion and mobility because of the presence of strong hydrogen bonding. This property limits their suitability for fast electrochemical reactions but proves advantageous in applications requiring controlled ion transport. Glyceline, on the other hand, shows conductivities between 0.1 and \SI{5}{\milli\siemens\per\centi\meter} and similarly exhibits high viscosity due to the polyol structure of glycerol, which promotes extensive hydrogen bonding. This viscosity can be mitigated by increasing the temperature, as the improved fluidity enhances ion mobility and as a consequence, glycerine conductive performance.115,118-120. However, despite the evaporation is minimal which benefits in supercapacitor applications, as can be observed in the CV test, both don't interact efficiently with the designed electrode because of their properties that make them less favourable for applications requiring rapid ion transport. Nonetheless, their characteristics, such as higher viscosity and slower ion mobility, make them well-suited for applications where controlled ion transport is preferred, such as in biocatalysis or controlled-release systems 115,118.


Conversely, in Figure 6(b) the sample BC6 (activated carbon post-EMS process) showed the highest specific capacity keeping the rectangular CV shape, showing a higher double-layer capacitance effect. BC6 electrochemical performance by CV was compared using ethaline 200, acetonitrile (I) and 0.6 M \ce{KOH} as an electrolyte. The results are described in Table~\ref{tab:specific_capacitances_bc6}. 

\begin{table}
\caption{Specific capacitance estimated by CV on Bco (black carbon)}
\label{tab:specific_capacitances_bc6}
\small
\centering
\begin{tblr}{
  hline{1-2} = {-}{},
  colsep = 7pt
}
Electrolyte &  {Specific\\capacitance (\unit{\farad\per\gram})} \\
ACN         &  17.79                                         \\
KOH         &  214.7                                        \\
Ethaline    &  110.16                                        \\
\end{tblr}
\end{table}

As can be observed in Figure 6(b) and Table~\ref{tab:specific_capacitances_bc6}, the interaction between BC6 as active material in the electrodes suggests and improved electrochemical behaviour with water-based electrolytes (\ce{KOH}) rather than organic electrolytes like I. I is commonly employed to widen the voltage window, improving energy density. However, the main reason why some organic solvents aren't able to act as efficient electrolytes is  due to their difference in polarity which varies the ion conductivity of the electrolyte121. The nature of activated carbon is abundant oxygen-containing functional groups (e.g., hydroxyl, carboxyl, and carbonyl groups), which are more hydrophilic. Organic solvents, specifically non-polar or weakly polar ones like acetonitrile or propylene carbonate, seem not to wet the activated carbon surface efficiently122. Building on this, there is a significantly effect on the poor wettability resulting in the inefficient electrolyte penetration of the electrolyte into the porous structure of activated carbon, reducing the available surface area for charge storage and dropping the overall capacitance of the SC. In contrast, some studies show that a mixture of alkaline or acidic chemicals can contribute to a better performance of supercapacitors, like diluted \ce{KOH}.  Favourable solvent-electrode interactions in water-based electrolytes also influence the stability and durability of the SC, as the water molecules facilitate the maintenance of the structural integrity of the activated carbon during charge-discharge cycling. Under this framework, the main challenge to overcome in the use of aqueous-based electrolytes is their narrow electrochemical stability window (~\SI{1.23}{\volt}). This constraint restricts the energy density of supercapacitors using aqueous electrolytes, in comparison with organic solvent-based systems, which can operate at higher voltages (2.5-\SI{3}{\volt})117. Nevertheless, KOH-based electrolyte plays an important role in the industry in total dependence of the applications especially those where high-power density and cycle life are prioritised.

On the other hand, is remarkable to mention that a specific capacitance of \SI{110.16}{\farad\per\gram} for ethaline, possesses a visible efficiency and stability with the BC6 electrodes. Polar nature of ethaline owing to the presence of both choline chloride and ethylene glycol, allows the formation of strong hydrogen bonds114,119. Activated carbon typically contains a variety of oxygen-containing functional groups on its surface, such as hydroxyl, carbonyl, and carboxyl groups. These groups facilitate the formation of hydrogen bonds or electrostatic interactions with the choline chloride and ethylene glycol components of ethaline. The activated carbon produced through EMS activation, as illustrated in Figure 6(d) using XPS analysis, displays distinct C1s and O1s peaks, indicating the presence of \ce{C=O} groups, particularly when molten salt activation is employed. Additionally, since Ethaline evaporates much more slowly compared to organic or water-based electrolytes, it can support extended cycling periods, as demonstrated in Figure 6I. 
It is remarkable to mention that the CV shows consistent behaviour over 150 cycles, closely resembling the trend observed up to 1000 cycles, corresponding to a capacitance retention of approximately 55.5\% (\SI{61.11}{\farad\per\gram}). Additionally, it is important to note that after 1000 cycles, supercapacitors typically experience failures due to electrolyte evaporation. However, tests with ethaline indicate that it retains a portion of the ELDC effect, valuable insights for further optimisation of long-life supercapacitors. This could potentially be achieved through alternative configurations, such as hybrid or asymmetric supercapacitors. The specific capacitance to a capacitance retention of approximately 44.7\%  after 5000 cycles (\SI{49.24}{\farad\per\gram}), highlighting the challenges of long-term stability and underscoring the need for improved electrolyte formulations or structural designs to mitigate capacitance loss123. Based on the measurements presented in Figure 6(d), at 150 cycles the maximum value of the specific capacitances was calculated to be around \SI{110.23}{\farad\per\gram} in the range of scan rate of \SI{100}{\milli\volt\per\second}, showing that ethaline electrolyte shows a well-defined, relatively rectangular-shaped CV curve. This indicates that the capacitor is operating with minimal resistance and exhibits a good electrochemical double layer capacitance (EDLC) behaviour, where the ions are able to move quickly to form a charge layer at the electrode surface. At higher scan rates, like \SI{50}{\milli\volt\per\second} and \SI{100}{\milli\volt\per\second}, the performance is more representative of practical, real-world supercapacitor behaviour, where faster charge/discharge is needed.

The current response in CV Is highly dependent on the scan rate, which can complicate the accurate extraction of specific capacitance, especially when non-ideal behaviours in or Faradaic reactions are involved, especially in this case was the interaction between a new material developed as BC6 using ethaline as electrolyte124-126. Galvanostatic Charge and Discharge (GCD) is suitable for evaluating energy storage devices' performance under practical charging conditions. Combining (GDC) with Electrochemical Impedance Spectroscopy (EIS) provides a powerful method to study the electrolyte-electrode interactions in electrochemical systems. EIS contribute to the understanding of key parameters like electrolyte resistance, double-layer capacitance, and charge transfer resistance124,125,127. GDC and EIS analysis were carried out as it is shown in Figure 8. 

[INSERT FIGURE 8: a) Galvanostatic charge-discharge profiles at different current densities of BC6 electrodes with ethaline as electrolyte at different current densities. b) EIS Nyquist Plot  and circuit modelling for BC6 and ethaline as electrolyte at room temperature.]

GDC electrochemical data were collected and calculated using the equation  The equation to calculate the specific capacitance $C_{sp}$ n Farads per gram (\unit{\farad\per\gram}) from electrochemical data for symmetric supercapacitor is described below 128: 
\begin{equation}
  C_{sp} = \frac{4 I}{m\cdot(\,dV~/~dt)}
\end{equation}
where $I$ is the constant current applied during the GCD process (\unit{\ampere}) and $m$ is the mass of the active material (\unit{\gram}). In order to evaluate the electrochemical performance of the BC6-ethaline supercapacitors (SCs), galvanostatic charge-discharge (GCD) measurements were conducted at different current densities. The current densities were calculated with the mass of active material mass loading of \SI{0.19}{\milli\gram\per\centi\meter\squared} and the voltage of \SI{1}{\volt} to avoid potential water electrolysis. The BC6 electrode demonstrated a near-ideal triangular profile with a specific capacitance of \SI{210.57}{\farad\per\gram}, indicating predominantly electric double-layer capacitance (EDLC) behaviour, as corroborated by the cyclic voltammetry (CV) results. Moreover, its longer discharge times at comparatively higher current densities, relative to other supercapacitors fabricated with activated carbon, highlight the influence of factors such as the source of the carbon, activation method, electrolyte concentration, and surface area. For instance, activated carbon derived from biomass can achieve specific capacitance values of approximately \SI{195}{\farad\per\gram} at a current density of \SI{0.5}{\ampere\per\gram} when activated via pyrolysis using 10\% \ce{KOH} as the activating agent 129. Similarly, another study reports that pyrolyzed black carbon from waste tyres exhibits a specific capacitance of \SI{212.2}{\farad\per\gram} at \SI{1}{\ampere\per\gram}, with a surface area of \SI{690.9}{\meter\squared\per\gram}, using a 1 M \ce{Et4NBF4}/\ce{AN} organic electrolyte 130. However, the presence of sulfur in the carbon generated during pyrolysis can limit its applicability in supercapacitors. Sulfur bonds are essential for maintaining the physical and chemical stability of rubber under environmental conditions, but breaking these bonds through devulcanization is both costly and energy-intensive. This process requires significant amounts of chemicals and energy, making it impractical for large-scale upcycling of black carbon from tyre pyrolysis, and thus it has not been widely adopted 131. 

On the other hand, developing carbon-based materials with high specific surface areas and optimised pore structures, particularly medium-sized pores (mesopores), is vital for improving the efficiency and longevity of energy storage systems. Post-recycling activation methods, such as potassium hydroxide (\ce{KOH}) impregnation, are frequently used to enhance the formation of mesopores. For instance, one study demonstrated an increase in surface area to \SI{115.2}{\meter\squared\per\gram}, achieving a specific capacitance of \SI{218}{\farad\per\gram} 132. Although this surface area nearly doubles that obtained for BC6 after EMS treatment, it is evident that, beyond porosity, factors such as electrode configuration and the interaction with electrolytes significantly influence supercapacitor performance131. 

The role of mesopores (2-\SI{50}{\nano\meter} in diameter) is particularly critical. These pores balance surface area and pore volume, enabling efficient ion transport and fast diffusion within the electrode. Mesopores provide a large surface area for ion adsorption, boosting overall energy storage capacity. At the same time, their size ensures easy access for electrolyte ions, enhancing the interaction between the electrolyte and the electrode material133. Furthermore, mesopores improve the structural stability of the electrode, reducing the risk of long-term degradation134. This balance between energy density and power density enables both high-charge storage and rapid energy delivery. Considering these attributes, the BC6 electrodes' performance in energy storage devices, specifically their energy density and power density, hinges on the interplay between material porosity, electrolyte interaction, and electrode architecture134-136.  

Considering these features, the BC6 electrodes demonstrate remarkable energy storage performance. Energy and power densities were calculated based on the total active mass of both electrodes in the symmetric device reaching an energy density of \SI{12.50}{\watt\hour\per\kilo\gram} and a higher power density of \SI{2.25}{\kilo\watt\per\kilo\gram} respectively. These results reflect lower energy density and a similar power density when compared to previously reported values for activated carbon electrodes derived from tyre waste in hybrid supercapacitors (e.g., \SI{4.7}{\watt\hour\per\kilo\gram} and a maximum power density of \SI{6362.6}{\watt\per\kilo\gram}). This difference underscores the critical role of supercapacitor configuration and the interactions between its components (especially influenced by the electrode-electrolyte interaction) in influencing overall device performance. This can be explained by a multi-level porous framework with extensively branched and interlinked voids reflected in the increased surface area of BC6 of the BET analyses and supported by the disorder of the graphitic structure presented in the XRD analysis and Raman spectroscopy. Pores at various length scales shorten the ion diffusion pathways, promoting increased diffusivity through the interlinked voids resulting in a more efficient electrode material-electrolyte electric double layer effect to fully utilize the electrode capacity, ultimately enhancing the energy density and power density135-137. 

On the other hand, EIS Figure 8 b) indicates excellent capacitive behaviour of the supercapacitor assembled with BC6 activated carbon by EMS. It can be observed by low charge transfer resistance, minimal solution resistance, and efficient ion diffusion, attributed to the porous structure of activated carbon. The presence of Warburg impedance (W) reflects the electrolyte diffusion at the electrode surface, confirming a positive interaction between the BC6- EMS electrochemically activated carbon and ethaline. Although high-energy-intensive methods, such as tyre pyrolysis or microwave activation, can produce specific surface areas between \SI{800}{\meter\squared\per\gram} and over \SI{2000}{\meter\squared\per\gram} (128 ) significantly higher than the ~\SI{60}{\meter\squared\per\gram} achieved by BC6 after EMS, however, the plot's proximity to the origin suggests and a steeper line of the circuit model in the curve, which signifies the capacitive behaviour of the SCs138. This fact reflects low solution resistance (Rs), characteristic of ideal capacitive behaviour, and a small semicircle in the high-frequency region, suggesting low charge transfer resistance. The equivalent circuit model provides an excellent fit, indicating that the observed impedance behaviour arises from a combination of solution resistance, charge transfer resistance, electric double-layer capacitance, and ion diffusion within the pores of the BC6 activated carbon139. At the same, the lower viscosity of ethaline (in comparison with the other DESs studied) and the molecular structure ensure good wettability of the electrode surface. This behaviour is characteristic of systems with efficient electrolyte diffusion across the surface of the supercapacitor, minimising resistance at the electrode-electrolyte interface134,137. The small semicircle observed in the charge transfer resistance (Rct) region indicates a highly efficient charge transfer process, which aligns with the energy storage mechanism in supercapacitors, dominated by electric double-layer capacitance rather than faradaic reactions and ensuring an intimate contact between the electrolyte and the electrode, minimising impedance at the electrode-electrolyte interface. The enhanced surface area of BC6 compared to the original Bco reduces overall system impedance by increasing capacitance in the low-frequency region and lowering charge transfer resistance 75,140,141.  

\section{Conclusions}
This study demonstrates the successful application of electrochemically activated carbon derived from plastic and tyre waste as a sustainable electrode material for supercapacitors. The electrochemical molten salts (EMS) process offers a promising approach for activating black carbon, a waste byproduct from tyre recycling, into high-performance carbon materials suitable for energy storage applications.  This study revealed a significant improvement in carbon properties post-EMS activation, as evidenced by a 39.4\% increase in BET-estimated surface area (from \SI{43.1}{\meter\squared\per\gram} to \SI{60.09}{\meter\squared\per\gram}) and enhanced pore characteristics, including a rise in pore volume and a shift in average pore size. These findings establish the EMS process as an effective and scalable method for producing porous carbon materials, laying the groundwork for optimising black carbon pre-treatment and activation protocols to maximize its electrochemical performance. The use of ethaline, a deep eutectic solvent, as an electrolyte for supercapacitors constructed with EMS-activated carbon demonstrates significant potential when compared to traditional water-based, organic, and other DES electrolytes. The specific capacitance of \SI{110.16}{\farad\per\gram} achieved with ethaline surpasses benchmarks for many conventional aqueous and organic systems, reflecting its superior ionic conductivity and compatibility with activated carbon. The study highlights the importance of further electrochemical testing and characterisation of ethaline and similar DES electrolytes to better understand their behaviour and interaction with EMS activated black carbon, considered a discarded product from recycling tyre waste. Such evaluations could unlock new pathways for developing sustainable supercapacitors and batteries, combining green electrolytes with waste-derived electrode materials to achieve high performance and environmental sustainability.

\section{Deep learning analysis}
\label{sec:ml_analysis}
Machine learning models can be thought of as universal function approximators. That is, provided enough data to learn from, they are able to model any continuous, closed domain function to an arbitrary degree of accuracy~\citep{geuchenUniversalApproximationComplexvalued2025}. Such models consist of interconnected ``neurons'' arranged in layers. These connections are dense, meaning each neuron is connected to all other neurons in the previous and next layer, giving rise to the common ``fully connected'' layer terminology. We will refer to a single layer as $F_{i}^{n,m}$ where $i$ is the position of the layer, $n$ is the input size (the number of input neurons expected), and $m$ is the output size. The layer computes:

\begin{equation}
\hat{y} = F_{i}^{n, m}(x) = W_{i}x + b_{i}
\end{equation}


where $x$ is an n-dimensional vector, $\hat{y}$ is an m-dimensional vector, $W_i$ is an $n \times m$ weight matrix and $b_i$ is a constant term.  Layers may be chained sequentially, as long as the inner dimensions of adjacent layers are equal, i.e. $W_i$ and $W_{i+1}$ can be multiplied. Such a sequence of connected layers is commonly referred to as a Multilayer Perceptron (MLP). An important element of the layers of the MLP is the activation function, which is applied to the output of each neuron, and introduces non-linearity into the model. This is essential for an MLP to fit the definition of a universal function approximator as defined earlier, because any sequence of fully connected layers must necessarily collapse into a single linear transform, which is not powerful enough to represent more complicated functions. The Rectified Linear Unit function (ReLU) is a popular activation function~\citep{agarapDeepLearningUsing2019} designed for this purpose and is defined as follows:

\begin{equation}
\text{ReLU}(x) = \begin{cases}
  0 & x \leq 0 \\
  x & x > 0
\end{cases}
\end{equation}

Thus, the MLP must have a ReLU activation function between each hidden layer, giving the final model structure as laid out in Figure~\ref{fig:model_arch}. Obtaining a $\hat{y}$ prediction from an MLP completes the ``forward pass'' part of training. The prediction is then evaluated against the target value $y$ (also called the ground truth value) using a cost function $L$ -- see Equations~\ref{eq:loss_current}~-~\ref{eq:loss} for how it is defined for this application. The partial derivative of $L$ is calculated with respect to each parameter $w_{i}^{n,m}$ in $W_i$: 

\begin{figure*}
 \centering
 \includegraphics[width=\linewidth]{model_arch.png}
 \caption{An overview of the training pipeline and model architecture, incorporating dropout after the largest layer to discourage overfitting. The input is a 6-dimensional vector consisting of a potential step, scan direction, and several experimental conditions under which the electrode material was sysnthesised; the output is a prediction of the current value at that potential. Four trainable weights $S^{4}$ provide a scaling factor for each of the conditions. Ground truth hysteresis curves shown in blue and model prediction shown in orange.}
 \label{fig:model_arch}
\end{figure*}

\begin{equation}
\frac{\partial L}{\partial w_{i}^{n,m}} = \frac{\partial L }{\partial \hat{y}} \cdot \frac{\partial \hat{y}}{\partial w_{i}^{n,m}}
\end{equation}

Computing all such partials completes the ``backwards'' pass of the training step -- a process known as backpropagation --  and gives us $\frac{\partial L}{\partial W_i}$, which tells the model how much, and the direction in which $W$ should change in order to minimise $L$. Repeating these forward and backwards passes over the MLP over and over again brings the weights closer to their ``ideal'' values -- ones for which $L$ is minimised. In other words, the optimsing algorithm uses the difference between the output of the model and its target to update $W$ in the correct direction. This is what is meant by training a machine learning model.  

\subsection{Previous work}
The application of machine learning systems to the prediction of specific capacitance values is varied -- some previous investigations involve predicting specific capacitance directly from experimental conditions~\citep{kumaryogeshMachineLearningApproach2025}, while others use these experimental conditions to predict hysteresis curves~\citep{ravichandranMachineLearningBasedPrediction2024}; the latter approach allows the indirect calculation of specific capacitance $C_{sp}$, which is proportional to the area between charge and discharge curves. It is this latter approach that is used by the present study, because predicting the full charge-discharge dynamics of cyclic voltammetry allows the model to generalise better to a wider range of conditions~\citep{deebansokCapacitiveTendencyConcept2024} -- especially important given the limited data available for this study. Using this approach, $C_{sp}$ is given by Equation~\ref{eq:specific_cap}, where $I(V)$ refers to the function that is approximated by the MLP.

\subsection{Dataset}
The complete dataset is made up of current samples from 5 cyclic voltammetry cycles -- see Table~\ref{tab:dataset}. These were collected via potentiostat with a scanrate of \SI{100}{\milli\volt\per\second}. Each cycle consists of 264 rows, with columns for current (I / \unit{\ampere}), voltage step (E / \unit{\volt}), active electrode mass (M / \unit{\milli\gram}), temperature (T / \unit{\degreeCelsius}), electrolysis duration (D / \unit{\hour}), electrolysis voltage ($V_e$ / \unit{\volt}), and scan direction (S). The last feature tells the model if the particular point is on the charge or discharge curve and is represented by either +1 (charging)  or -1 (discharging).  

\begin{table}
\caption{Experimental conditions that were used during the synthesis of the electrodes for each cycle that constitutes the dataset used for training our model. The scanrate was \SI{100}{\milli\volt\per\second} in all cases.}
\label{tab:dataset}
\small
\centering
\begin{tblr}{
  hline{1-2} = {-}{}
}
Cycle      & M / mg  & T / $^{\circ}$C& D / h & $V_e$ / V & Electrolyte \\
BCMS2      & 0.11    & 800            & 2     & 2         & Ethaline    \\
BCMS3      & 0.11    & 750            & 2     & 2         & Ethaline    \\
BCMS5      & 0.18    & 650            & 2     & 2         & Ethaline    \\
BCMS7      & 0.25    & 700            & 2     & 2.8       & Ethaline    \\
BCMS9      & 0.25    & 750            & 2     & 2.8       & Ethaline       
\end{tblr}
\end{table}

Notice that the various inputs span many orders of magnitude -- this is a problem for machine learning models. Using raw features such as those presented in Table~\ref{tab:dataset} mean that ones with large absolute values dominate the weights of the model, even though they may not be as important, and vice versa, which can lead to lower performance~\citep{kimInvestigatingImpactData2025}. For this reason, all features are scaled across all cycles such that they have zero mean and unit standard deviation. To transform an individual instance of a feature $d_i$ (e.g. electrode mass, temperature) to its scaled value $d_{norm,i}$:

\begin{equation}\label{eq:norm}
  d_{norm,i} = \frac{d_i - \mu(D)}{\sigma(D)}
\end{equation}

where $D$ is the collection of all the values for this feature, $\mu$ is the mean of values in $D$, and $\sigma$ is their standard deviation. These scaled values are used in all subsequent machine learning operations, and final outputs are de-normalised using the inverse of Equation~\ref{eq:norm} for display and comparison with experimental values. Current values were also transformed in this way. However, the potential step $V$ was rescaled slightly differently: rather than ensuring zero mean and unit variance, we rescale them such that the minimum potential across the whole dataset is -1 and the maximum is +1:

\begin{align}
  V_{mid} &= \frac{V_{max} + V_{min}}{2} \nonumber \\
  V_{half-range} &= \frac{V_{max} - V_{min}}{2} \nonumber \\
  V_{norm} &= \frac{E - V_{mid}}{V_{half-range}}
\end{align}

where $V_{min}$ and $V_{max}$ are the minimum and maximum potential values \emph{across the whole dataset}. This scaling ensures that all cycles share a common and bounded potential domain, making it easier for the model to lear consistent relationships across different experimental conditions.

The input to the model is therefore a 6 dimensional, normalised feature vector that consists of a potential step and experimental conditions under which the electrode material was synthesised, and the output is the current value at the potential step.  

\subsection{Experimental design and model architecture}
\label{subsec:arch}
We employ leave-one-out cross validation when training: for any one run, the model is trained only on points from 4 of the cycles; points from the 5$^{th}$ are held out for validation. Here, validation means only evaluating the forward pass of the model and not updating the weights of the model. It is a test to determine if the model is able to generalise to ``unseen'' data, and each cycle is subjected to this treatment in turn. We repeat training runs five times independently to build confidence that the model's output is reliable, and report the average in all subsequent results.  

Due to the small size of the dataset, we introduce a simple heuristic to improve the training stability and to encourage the model to learn meaningful relationships between experimental conditions and the voltammetric response. This is realised by introducing four trainable parameters that act as scaling factors for the experimental conditions that are used when synthesising the black carbon -- active mass (M), electrolysis temperature (T), electrolysis duration (D) and electrolysis voltage ($V_e$). These parameters allow the model to explicitly learn the relative importance of each condition rather than solely relying on implicit weighting within the first layer of the MLP. 

As illustrated in Figure~\ref{fig:model_arch}, the learned scaling parameters $S^{4}$ are applied to the condition vector prior to concatenation with the potential step (E) and scan direction (S), resulting in the 6-dimensional input that is expected by the MLP. Scaling is applied in the following way: the learned weights are first divided by a temperature parameter $\tau$, which controls the sharpness of the relative weighting, and subsequently normalised by the softmax function~\citep{bridleTrainingStochasticModel1989} to give the scaling factor $f_i$ for each learned weight:

\begin{equation}
  \label{eq:softmax}
  f_i = \frac{\exp(s_i / \tau)}{\sum_{j=1}^{4} \exp(s_j / \tau)}
\end{equation}

where $s_i$ is any one of the four learned weights. The resulting normalised weights are multiplied element-wise by the condition vector, yielding the weighted condition vector.

The rest of the model is a multilayered perceptron with 4 hidden layers in between a 6-neuron input layer and a 1-neuron output layer. Please refer to Figure~\ref{fig:model_arch} for the full model architecture. Note the dropout layer, which randomly (with some probability $p$) deactivates a subset of neurons during training. This is done so that the model does not rely too heavily on any single neuron or set of neurons, thereby reducing overfitting and improving generalisation to new data~\citep{mouDropoutTrainingDatadependent2018}. Wider layers with more neurons benefit more from dropout as they are more at risk from overfitting~\citep{srivastavaDropoutSimpleWay2014}, hence the placement between $F_{2}$ and $F_{3}$.

As mentioned in the Section~\ref{sec:ml_analysis}, the objective function scores the output of the MLP so it can be used in gradient calculations, which in turn are used to inform updates to the model's weights. Because the model predicts the current at a particular voltage step, part of the loss function is the mean squared error (MSE) between the predicted and actual value. This is given by:

\begin{figure*}
 \centering
 \includegraphics[width=\linewidth]{results.png}
 \caption{A plot reporting the average normalised $C_{sp}$ over 5 runs per cycle per experimental setting. Error bars indicate 95\% confidence intervals over runs.}
 \label{fig:results}
\end{figure*}

\begin{equation}
  \label{eq:loss_current}
  L_{curr} = \frac{1}{N} \sum_{i=1}^{N} (\hat{y}_i - y_i)^{2}
\end{equation}

where $\hat{y}_i$ is the predicted current value, $y_i$ is the true value, and $N$ is the number of predicted points. Further, in order to maintain a smooth curve that is physically consistent with the behaviour of real hysteresis curves, we impose a penalty on the second derivative of the predicted current with respect to potential, discouraging sharp local curvature and jitter:

\begin{equation}
  \label{eq:loss_smooth}
  L_{smth} =  \Bigl\lVert\frac{\partial^2 I_{pred}}{\partial E^2}\Bigr\rVert_{2}^{2}
\end{equation}

where $I_{pred}$ is the predicted current and $E$ is the potential. Finally, we also use the area of the predicted curve in another MSE loss calculation to encourage the model to produce the correct net area exhibited by the experimental curve:

\begin{equation}
  \label{eq:loss_area}
  L_{area} = \frac{1}{M} \sum_{j=1}^{M} (\hat{a}_j - a_j)^{2}
\end{equation}
where $M$ is the number of cycles, $a_j$ is the actual area between the curves for a given cycle, and $\hat{a}_j$ is the area calculated using predicted current values.  The losses are combined for final  objective for the MLP:
\begin{equation}
  \label{eq:loss}
  L = L_{curr} + L_{smth} + L_{area}
\end{equation}


In order to assess the importance of these loss functions (as well as other  model settings), an ablation study was conducted to measure the impact these settings on the final prediction. A comparison of $C_{sp}$ predicted by models under these experimental settings is given in Figure~\ref{fig:results}, and a comparison of their relative errors is given in Table~\ref{tab:errors}. The base MLP, henceforth called ``Base'' is only tasked with the $L_{curr}$ objective. The following experimental modifications were added to the Base model:
\begin{enumerate}
  \item  ``+ D'' -- inclusion of a \textbf{D}ropout layer, as discussed in Section~\ref{subsec:arch}
  \item  ``+ S'' -- addition of the \textbf{S}moothing $L_{smth}$ objective to $L$
  \item  ``+ A'' -- addition of the \textbf{A}rea $L_{area}$ objective to $L$ 
  \item  ``+ W'' -- inclusion of trainable \textbf{W}eights for each of the synthesis conditions 
\end{enumerate}


\subsection{Results and discussion}


\begin{table}
\caption{Average error (relative \%) for each cycle in leave-one-out cross validation over 5 independent runs. Base+D+A exhibits the least overall error across all the settings.}
\label{tab:errors}
\small
\centering
\begin{tblr}{
  hline{1-2,7,8,9} = {-}{},
  colsep = 4pt
}
Cycle   & Base    & {Base\\+D}  & {Base\\+D\\+S}  & {Base\\+D\\+S\\+A}   & {Base\\+D\\+A} & {Base\\+D\\+A\\+W}\\
BCMS2   & 0.62    & 0.58        & 2.54            & 7.17                 & 0.77           & 1.22              \\
BCMS3   & 29.82   & 23.28       & 20.15           & 16.11                & 16.93          & 1.53              \\
BCMS5   & 7.04    & 15.05       & 7.88            & 2.61                 & 8.83           & 1.77              \\
BCMS7   & 30.87   & 25.37       & 18.50           & 26.78                & 17.92          & 7.42              \\
BCMS9   & 9.12    & 4.64        & 1.91            & 2.42                 & 0.39           & 6.83              \\
Average & 15.49   & 13.79       & 10.20           & 11.02                & 8.97           & \textbf{3.75}
\end{tblr}
\end{table}

Results were collected for each experimental setting described in Section~\ref{subsec:arch}. Figure~\ref{fig:results} demonstrates the averaged, normalised $C_{sp}$ values computed over five independent runs. Predicted $C_{sp}$ were normalised by their corresponding ground-truth values to facilitate comparison across settings, as $C_{sp}$ spans several orders of magnitude. Error bars indicate the 95\% confidence interval over the five runs. Tables \ref{tab:errors} and \ref{tab:cond_weights} report results for each leave-one-out fold, where the Cycle column indicates the curve that was left out.

Overall, the Base+D+A+W configuration (corresponding to the model incorporating dropout, condition scaling, and the area loss of Equation~\ref{eq:loss_area}) exhibits the samllest deviation from the $C_{sp}$ baseline, on average.  This setting also yields comparatively narrower confidence intervals, suggesting improved stability across runs. Incorporating the area-based loss term resulted in improved performance across all curves. As shown in Figure~\ref{fig:results}, models that include this loss term tend to produce predictions that are closer to the target baseline. It should be noted that the smoothness loss term (Equation~\ref{eq:loss_smooth}) seems to be somewhat at odds with the area loss. This effect is reflected in  Table~\ref{tab:errors}, which presents the relative error percentage between the predicted and true $C_{sp}$ values across all the experimental settings. The addition of area loss to Base+D+S (10.2\% error) results in the error increasing slightly to 11.02\%. Removing the smoothness loss (corresponding to the Base+D+A configuration) yields a lower overall error (8.97\%). This suggests that the two loss terms may impose competing constraints on the learned representations, which could partially limit their combined effectiveness. However, the most effective strategy incorporates condition scaling to the latter configuration (giving Base+D+A+W), and yielding an average error of 3.75\%.

\begin{table}
\caption{Relative weights assigned by the proposed model to the various molten salt parameters. The exclusion of one curve from each training run under our leave-one-out regime gives a slightly different set of weights for each experiment.  Overall, the model was most sensitive to temperature (T), electrolysis voltage ($V_e$) and active mass (M). Electrolysis duration (D) was found to be the least important input in predicting $C_{sp}$.}
\label{tab:cond_weights}
\centering
\begin{tblr}{
  hline{1,2,7,8} = {-}{},
}
Cycle   & M     & T     & D     & $V_e$ \\
BCMS2   & 0.193 & 0.294 & 0.173 & 0.339 \\
BCMS3   & 0.266 & 0.264 & 0.237 & 0.233 \\
BCMS5   & 0.254 & 0.285 & 0.175 & 0.286 \\
BCMS7   & 0.223 & 0.301 & 0.202 & 0.275 \\
BCMS9   & 0.214 & 0.413 & 0.170 & 0.203 \\ 
Average & 0.230 & 0.311 & 0.191 & 0.267 \\
\end{tblr}
\end{table}

The inclusion of trainable scaling values for the four experimental conditions (M, T, V and D) offers insights into the relative importance of each of these in determining $C_{sp}$. Table~\ref{tab:cond_weights} shows  demonstrates that on average, the model assigns most importance (31.1\% relative weight) to the temperature of the molten salt (T). This is followed by electrolysis voltage ($V_e$) at 26.7\%, and active material mass (M) at 23.0\%. Finally, duration of electrolysis (D) recieves the lowest relative weight (19.1\%). This is unsurprising, as this setting did not vary at all  across the dataset (see Table~\ref{tab:dataset}), so could not explain any of the variance in the predicted output.

A key limitation of the experimental setup lies in the limited diversity of the input features. Although each of the five cycles contains 264 samples, all samples within a given cycle share the same experimental conditions; the only varying factor is the voltage step. Moreover, the voltage progression itself is identical across all cycles. Thus, even though a unique feature vector exists for each point, the dataset comprises only 5 unique condition vectors across 1320 samples. This extremely low condition variance substantially constrains the expressive capacity available to the model and limits the generalisability of the learned representations. In this context, the fact that the model exhibits meaningful performance at all is somewhat surprising, and underscores the need for more diverse experimental conditions (i.e., a wider range of these conditions) in future studies.

\section{Conclusion}
The conclusions section should come in this section at the end of the article, before the Author contributions statement and/or Conflicts of interest statement.


%% The Appendices part is started with the command \appendix;
%% appendix sections are then done as normal sections
%% \appendix


% To print the credit authorship contribution details
% \printcredits

%% Loading bibliography style file
% \bibliographystyle{model1-num-names}
\bibliographystyle{cas-model2-names}
% Loading bibliography database
\bibliography{refs}

% Biography
%\bio{}
% Here goes the biography details.
%\endbio

%\bio{pic1}
% Here goes the biography details.
%\endbio

\end{document}

